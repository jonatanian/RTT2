\section{Presentación}

El presente documento habla del Trabajo Terminal numero TT2014-B060 realizado en el periodo 2015-2016/1 por los alumnos Alcántara Carrillo Oscar, Castañeda Chavero Jonatan Ian, Ruiz González Brenda Angélica en la Escuela Superior de Cómputo en la Ciudad de México, a los 20 días del mes de Noviembre del 2015.\\

Va dirigido a directores, sinodales, el personal del CMPL y cualquier otra persona interesada en el control de correspondencia como: alumnos, egredasos, docentes, administrativos, etc.\\
La finalidad es que se pueda evaluar el trabajo realizado, sus limitantes, los retos encontrados, logros y conclusiones.\\

Este documento no contiene la norma sobre control de correspondencia sin embargo, sí contiene un análisis de ella y una referencia además, una descripcion del desarrollo de este proyecto. Contiene información sobre el CMPL, la problemática planteada en TT1, objetivo, avances de TT1 y alcance. \\

\subsection{Organización}

La organización de este documento está de la siguiente manera: un total de tres capítulos, en el capítulo 1 se presenta los antecedentes para está segunda parte que son el contexto, control de correspondencia, la problemática planteada, objetivos, avances y el alcance del proyecto. En el capítulo 2 se presenta el trabajo realizado es decir, cómo se implementó para está segunda evaluación y por último el capítulo 3 los resultados objetidos y conclusiones. \\
