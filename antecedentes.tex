\section{Contexto}

El Centro Mexicano para la Producción más Límpia (CMPL) es una instancia del Instituto Politécnico Nacional (IPN), que se dedica a la investigación y elaboración de procesos de producción más limpia. El CMPL, como dice en su página web:

	\begin{quotation}Es integrante de la red mundial de centros de producción más límpia y de la red latinoamericana de producción más limpia, promovidas por la Organización de las Naciones Unidas para el Desarrollo Industrial (ONUDI) y el Programa de las Naciones Unidas para el Medio Ambiente (PNUMA). Cuenta con 13 años de experiencia realizando trabajo técnico para la industria nacional atendiendo sectores como alimentos, petroquímicos, cementeros, galvanoplastia y embotelladoras, por mencionar algunos. Los servicios que ofrece el CMPL son: diagnóstico en producción más limpia y eficiencia energética, diplomados a distancia, presenciales y Maestría en Producción Más Limpia, realización de proyectos de mecanismo de desarrollo limpio, planes de manejo de residuos y análisis de químicos \cite{PoliticaCMPL}.\\
	\end{quotation}
	
Para lograr lo anterior, el CMPL cuenta con los siguientes objetivos:
	
	\begin{itemize}
		\item Llevar a cabo un proceso de mejora continua en todos los ámbitos a través del establecimiento y revisión de objetivos y metas \cite{PoliticaCMPL}.
		\item Tener en cuenta los requisitos establecidos por nuestros clientes \cite{PoliticaCMPL}.
		\item Asumir el compromiso de cumplir los requisitos aplicables, tanto legales y reglamentarios como otros que la organización suscriba \cite{PoliticaCMPL}.
		\item Implicar, motivar y comprometer a todo el personal para que se involucre en la organización, así como su formación, motivación y comunicación \cite{PoliticaCMPL}.
		\item Desarrollar actividades formativas para que todos los integrantes del CMPL conozcan, participen y apliquen el Programa de Protección Civil del IPN \cite{PoliticaCMPL}.
		\item Establecer como uno de nuestros objetivos principales la prevención de la contaminación \cite{PoliticaCMPL}.
		\item Utilizar de modo racional, oportuno, pertinente y adecuado los recursos materiales, fomentar el ahorro energético y la reducción de la producción de residuos\cite{PoliticaCMPL}. 
	\end{itemize}
	
	Como apoyo para la realización de dichos objetivos, el CMPL tiene objetivos de calidad y del medio ambiente bien definidos. Con lo cuál requiere estar certificado por dos normas (ISO 9001:2008 y 14001) y contar con un Sistema Integral de Gestión para la Calidad y el Medio Ambiente (SIG). Además de que existe otra norma aplicables a considerar que es el Manual de Procedimientos del Centro Mexicano para la Producción más Limpia; Este es un instrumento que se basa en el manual de procedimientos tipo de los Centros de Investigación, para su realización se contó la asesoría de la Dirección de Planeación. El propósito del manual es promover la realización ordenada y eficiente de las actividades del Centro Mexicano para la Producción más Limpia con el fin de ofrecer servicios ágiles, efectivos, con el firme objeto de mantener al día sus esquemas de operación y control\cite{ManProcCMPL}.\\

Para fines de este proyecto, nos enfocaremos en el procedimiento para el control y gestión administrativa de documentos del CMPL, en adelante Control de Correspondencia que comprende: oficios entrantes, salientes y de respuesta y memorándums. Dicho procedimiento se menciona a continuación. \\
	
\section{Control de correspondencia}

El propósito de este procedimiento es realizar el control y gestión de los documentos que ingresan y se generan al interior de la unidad responsable, para brindar atención expedita y oportuna a las instancias que lo requieran. Este procedimiento es de aplicación generalizada y obligatoria para todo el personal que tiene asignada alguna actividad en el control y gestión administrativa de documentos en el Centro Mexicano para la Producción más Limpia\cite{ManProcCMPL}.\\
El procedimiento que marca este manual para el control de correspondencia es el siguiente: \\

\begin{enumerate}
		\item Recibe documentos y revisa que cumpla con los requisitos establecidos en las políticas de operación. ¿Cumple con los requisitos?
		\item No, regresa documentos y/o información. Pasa a la actividad 1.
		\item Si, sella de recibido de la Dirección indica sus iniciales de quién recibió el documento y hora de recepción.
		\item Registra documentos en el apartado de control de correspondencia del intranet y registra si contienen anexos y o archivos magnéticos, posteriormente los pasa a revisión de la dirección.
		\item Asigna y turna al órgano del CMP+L que dará atención al documento.
		\item Recibe y atiende indicaciones. ¿Requiere respuesta?
		\item No, archiva y pasa al fin del procedimiento.
		\item Sí, elabora documento de respuesta.
		\item Recibe y analiza documento de respuesta. ¿Cumple con las indicaciones?
		\item  No, realiza observaciones. Pasa a la actividad 6.
		\item Si, Autoriza, firma y lo turna al órgano que atendió.
		\item  Envía documento de respuesta con su respectivo acuse a oficialía de partes para su entrega ante la instancia correspondiente.
		\item Entrega oficio o memorándum firmado y/o documento de respuesta a la instancia correspondiente y se asegura de que el acuse sea debidamente sellado.
		\item Recibe acuse de recibido y archiva.
		\item Termina proceso.
	\end{enumerate}
	
Una vez analizado el procedimiento anterior que se lleva actualmente y la situación del CMPL se tiene la siguiente problemática. 

\section{Problemática}

Al no contar con un sistema que les apoye en la realización del procedimiento anterior todo el personal del CMPL se ve en la necesidad de llevar ese proceso de forma manual lo que ocasiona que haya problemas con: \\

\subsubsection{Los oficios y memorándums}

Tales como: 
\begin{itemize}
	\item No siempre se le informa al destinatario la llegada de sus oficios o memorádums.
	\item Los oficios con información confidencial es propensa a estar expuesta (violación de la privacidad).
	\item No se sabe el estatus del seguimiento de los oficios.
	\item Es complicado consultar la información de los oficios que están archivados.
	\item El registro de los oficios no se realiza de forma correcta.
	\item Rezago tecnológico.
	\item Se atienden oficios fuera de la fecha límite de respuesta.
	\item No se sabe el desempeño de trabajo del centro respecto a la atención de asuntos con dependecias externas.

	\item Es complicado saber la ubicación física de algunos oficios o memorándums. %Explicación
	\item Varios de los oficios no son revisados y validados por la oficialía de partes.
\end{itemize}
Y problemas con: 

		\subsubsection{El manejo de la información}
\begin{itemize}
	
	\item Los empleados del CMPL no siempre se pueden poner en contacto con el personal indicado.		
	\item Es complicado consultar las evidencias fotográficas en orden cronológico.
	\item El personal no siempre consulta la información actualizada de los cursos.
	\item No todo el personal se entera de las actividades que se llevarán a cabo en el CMPL.
	\item Se vuelve complicado obtener el material de apoyo.		
\end{itemize}

De acuerdo con lo anterior, para fines de este proyecto sólo se abordarán los problemas con los oficios y memorándums para ello se plantea el siguiente objetivo. \\

\section{Objetivo del proyecto}

Para mitigar las causas y reducir el impacto de los problemas que se tienen nos basamos en el siguiente objetivo: \\

Desarrollar e implementar una aplicación web que apoye a las tareas de recepción y entrega de correspondencia interna y externa del CMPL. \\

\subsection{Objetivos Específicos}

\begin{itemize}
	\item Ayudar al personal del CMPL a cumplir los líneamientos establecidos. %cuales 
	\item Que la aplicación apoye a la consulta de correspondencia entrante y saliente.
	\item Que la aplicación envíe notificaciones al destinatario.
	\item Que la aplicación apoye a ver el seguimiento de los asuntos que se deben atender.
\end{itemize}

Dicho lo anterior se presentaron los siguientes avances. 

\section{Avances}
Identificación de la problemática, las causas e impacto, el procedimiento actual, así como una mejora al procedimiento, además de ...
%que se presentó a donde llegamos y a que nos comprometimos 

\section{Alcance del proyecto}


\subsection{Requetimientos Funcionales}
\begin{itemize}
	\item RF1 El personal podrá autenticarse en la aplicación web.
	\item RF2 El usuario podrá cerrar sesión en la aplicación web.
	\item RF3 El usuario podrá registrar correspondencia. 
	\item RF4 El usuario podrá turnar la correspondencia. 
	\item RF5 El usuario podrá compartir copia de la correspondencia. 
	\item RF6 La aplicación web notificará al usuario cuando haya recibido nueva correspondencia.
	\item RF7 El usuario podrá recibir correspondencia.
	\item RF8 La aplicación web enviará un correo electrónico de notificación de correspondencia entrante.
	\item RF9 El usuario podrá consultar la correspondencia. 
	\item RF10 El usuario podrá asignar prioridad a la correspondencia cuando se registre.
	\item RF11 El usuario podrá cancelar el proceso de la correspondencia. 
	\item RF12 El usuario podrá descargar una copia digital de la correspondencia. 
	\item RF13 El usuario podrá guardar en la aplicación web una copia digital de la correspondencia.
	\item RF14 El usuario podrá registrar usuarios en la aplicación web. 
	\item RF15 El usuario podrá dar de baja usuarios en la aplicación web. 
	\item RF16 El usuario podrá modificar los datos personales de los usuarios.
	\item RF17 El usuario podrá consultar a los trabajadores registrados en la aplicación.
	\item RF20 El usuario podrá cambiar su contraseña. 
	\item RF21 El sistema generará números de oficios consecutivos. 
	\item RF22 El usuario podrá restablecer la contraseña.
\end{itemize}
 

\subsection{Requerimientos No Funcionales}

\begin{itemize}
	\item La aplicación web deberá ser compatible con cualquier navegador.
	\item RNF2 El archivo de correspondencia digitalizado no deberá pesar más de 1MB.
	\item RNF3 La aplicación web deberá estar idealmente disponible en el horario hábil laboral.
	\item RNF4 La aplicación web deberá ser de fácil uso.
	\item RNF5 La aplicación web deberá ser adaptable a futuras modificaciones.
	\item RNF6 La aplicación web deberá contener vistas con colores en tonalidades verdes.
	\item RNF7 La aplicación web será responsivo para dispositivos móviles de Apple y Android.
	\item RNF8 La aplicación sólo aceptará guardar archivos .doc, .docx y .PDF
	\item RNF9 Los archivos anexos no deberán pesar más de 10 MB.
	\item RNF10 La aplicación web deberá ser implementada en la intranet del CMPL.
\end{itemize}

\subsection{Reglas de Negocio}

\subsection{Proceso}
Pricipales mejoras: \\ 

\begin{itemize}
	\item Alertas al usuario: cuando el usuario este utilizando la aplicación está mostrará los asuntos que no ha atendido. 
	\item Notificaciones al correo: una vez que se turne un oficio, se enviará un correo electrónico al usuario diciendo que tiene un nuevo asunto por atender. 
	\item Respaldos: el administrador podrá realizar los respaldos de todos los registros de la corespondencia así como los documentos digitales.
	\item Seguimiento: el seguimiento de la correspondencia se refiere a si ya fue atendido, si está en atención, si fue respondido o si está enterado.
\end{itemize}

Para fines de este proyecto sólo se considerará la confiabilidad, la disponibilidad, la robustes y la seguridad debido a que realizar un sistema con todas las propiedades no funcionales resulta muy complejo y requeriría más tiempo del que se dispone para lograse, además de que se acordó con el CMPL que no requiere que lleve todas las propiedades porque el sistema es sólo para operarse dentro del centro. A continuación se define cada una de ellas.

\subsection{Propiedades de Software}

\textbf{Seguridad\\}
Indica la capacidad de un sistema de software para evitar fallos que se traducirá en la pérdida de vidas, lesiones, daños significativos a la propiedad o destrucción de la propiedad\cite{Seguridad}.

\textbf{Confiabilidad\\}
La confiabilidad de un sistema es la probabilidad de que el sistema realizará su funcionalidad bajo los limites de diseño especificados, sin fallo, durante un periodo de tiempo determinado\cite{Seguridad}.\\

\textbf{Disponibilidad\\}
La disponibilidad es la probabilidad de que el sistema esta operando en un tiempo particular\cite{Seguridad}.\\ 

\textbf{Robustes\\}
Un sistema de software es robusto si es capaz de responder adecuadamente a las condiciones de tiempo de ejecución no anticipados\cite{Seguridad}.\\

\subsection{Plataforma}
Para que la aplicación funcione se necesita de los componentes de hardware, software y los servicios. \\

\subsubsection{Hardware}

Dentro del hardware el CMPL actualemnte cuenta con:
\begin{enumerate}
	\item Un servidor que de el servicio a las computadoras, aloje la aplicación así como los registros de la correspondencia y se hagan los respaldos. Este servidor deberá contar con: 
	\begin{itemize}
		\item Al menos con 500 GB de disco duro.
		\item Un monitor.
		\item Un teclado.
		\item Un mouse.
		\item Al menos 8 GB en memoria RAM.
		\item Un procesador Intel Xeon.
	\end{itemize}
	\item Cada trabajador deberá contar con una computadora ya sea de escritorio o laptop con: 
	\begin{itemize}
		\item Al menos 256 GB de disco duro.
		\item 2 GB en memoria RAM.
		\item Un procesador Intel Quad Core. 
	\end{itemize}
\end{enumerate}

\subsubsection{Software}
El software que requiere es: 
\begin{enumerate}
	\item Un sistema operativo tanto para el servidor como las computadoras.
	\item Un navegador de internet.
	\item Un gestor de base de datos.
	\item Un servidor de correos.
	\item Un servidor web.
\end{enumerate}

\subsubsection{Servicios}

Por último se requiere de los siquientes servicios: 
\begin{itemize}
	\item Energía eléctrica para el funcionamiento
	\item Una planta de luz para que cuando se vaya la luz el servidor siga operando.
	\item Un sistema de enfriamiento para refrigeración del servidor y evitar averias.
	\item Site para colocar el servidor.
	\item Una puerta con llave para que ninguna persona diferente al jefe del departamento de sistemas y banco de datos entre y modifique la configuración del servidor.
	\item Que el administrador genere respaldos.
	\item Cámaras de seguridad. 
	\item Servicio web para que los usuarios accedan a la aplicación. 
	\item Configuración de red.
\end{itemize} 

\subsection{Interacción con el Usuario}
La interacción de la aplicación con el usuario será mediante pantallas con formularios, mensajes de confirmación, alertas. Para esto se utlizarán herramientas como JQuery o Ajax porque nos permite simplificar la manera de interactuar con los documentos HTML, manipular las hojas de estilo para dar efectos.

\subsection{Modelo de Información}

La base de datos 